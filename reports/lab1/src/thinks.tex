\section{Выводы}

В ходе выполнения первой лабораторной работы по курсу \enquote{Информационный поиск} я познакомился с процессом формирования текстового корпуса из реальных веб-источников. На практике были изучены особенности работы с «сырыми» HTML-данными и сложности, возникающие при извлечении полезного текстового содержимого.

В процессе выполнения работы стало понятно, что даже структурированные сайты содержат большое количество служебной информации, не относящейся к основному тексту, и требуют аккуратной очистки и обработки. Также был получен опыт выделения мета-информации документов, такой как авторы и заголовки, которая может быть полезна при дальнейшем анализе и поиске.

Дополнительно была рассмотрена работа существующих поисковых систем и выявлено, что, несмотря на их удобство для конечных пользователей, они имеют ограничения с точки зрения исследовательских задач информационного поиска. Это подчёркивает необходимость создания собственных поисковых решений для проведения экспериментов и анализа.

Полученные навыки будут полезны при выполнении последующих лабораторных работ, а также в задачах анализа текстов, построения поисковых систем и обработки больших объёмов неструктурированных данных.
