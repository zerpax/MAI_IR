\section{Описание}

В рамках лабораторной работы был подготовлен корпус документов для дальнейшего использования в задачах информационного поиска. В качестве источников данных были выбраны сайты \textit{Billboard.com} и \textit{Pitchfork.com}, представляющие собой крупные англоязычные онлайн-издания, публикующие новости, обзоры и аналитические материалы, посвящённые музыкальной индустрии.

Данные сайты были выбраны по следующим причинам: регулярное обновление контента, наличие большого количества текстовых материалов, чёткая структура статей, а также существование встроенных и внешних средств поиска, что делает корпус пригодным для выполнения последующих лабораторных работ.

Для формирования корпуса HTML-страницы были разобраны и очищены от служебной информации. Из документов были удалены элементы навигации, рекламные блоки, скрипты и иные части страницы, не относящиеся к содержанию статьи. В результате из каждого документа был выделен чистый текст, включающий заголовок, авторов и основной текст статьи.


Для выбранных источников существуют готовые поисковые системы. Оба сайта обладают встроенным поиском по опубликованным материалам. Кроме того, поиск по данным корпусам возможен с использованием внешних поисковых систем, таких как Google и Яндекс, с применением ограничений на домен (например, \texttt{site:billboard.com} или \texttt{site:pitchfork.com}). Таким образом, корпус удовлетворяет требованиям лабораторной работы и может быть использован в дальнейших заданиях.

Для выбранного корпуса существуют готовые поисковые системы, которые могут быть использованы для поиска по документам. Оба сайта-источника, \textit{Billboard.com} и \textit{Pitchfork.com}, обладают встроенными средствами поиска по опубликованным материалам. Кроме того, поиск по данным ресурсам возможен с использованием внешних поисковых систем, таких как Google и Яндекс, с применением ограничений на домен.

В качестве примера был рассмотрен поисковый запрос \enquote{Justin Bieber}. На рисунках~\ref{fig:google_search} и~\ref{fig:yandex_search} представлена поисковая выдача Google и Яндекса соответственно, полученная с использованием ограничений на сайт музыкальных изданий. На рисунках~\ref{fig:billboard_search} и~\ref{fig:pitchfork_search} показаны результаты встроенного поиска сайтов Billboard и Pitchfork.

\begin{figure}[h!]
    \centering
    \includegraphics[width=0.85\textwidth]{google.png}
    \caption{Результаты поиска запроса \enquote{Justin Bieber} в поисковой системе Google.}
    \label{fig:google_search}
\end{figure}

\begin{figure}[h!]
    \centering
    \includegraphics[width=0.85\textwidth]{yandex.png}
    \caption{Результаты поиска запроса \enquote{Justin Bieber} в поисковой системе Яндекс.}
    \label{fig:yandex_search}
\end{figure}

\begin{figure}[h!]
    \centering
    \includegraphics[width=0.85\textwidth]{billboard.png}
    \caption{Результаты встроенного поиска сайта Billboard.com по запросу \enquote{Justin Bieber}.}
    \label{fig:billboard_search}
\end{figure}

\begin{figure}[h!]
    \centering
    \includegraphics[width=0.85\textwidth]{pitchfork.png}
    \caption{Результаты встроенного поиска сайта Pitchfork.com по запросу \enquote{Justin Bieber}.}
    \label{fig:pitchfork_search}
\end{figure}


В результате выполнения работы была получена следующая статистическая информация о корпусе:
\begin{itemize}
    \item Размер сырых данных: примеры HTML-документов имеют размер около 1051~КБ для Billboard и около 715~КБ для Pitchfork.
    \item Общее количество документов: около 30\,000.
    \item Размер выделенного текста: для примеров документов Billboard объём очищенного текста составляет 5~КБ, Pitchfork - 3~КБ.
    \item Средние значения размера документа и объёма текста в документе будут рассчитаны на последующих этапах работы.
\end{itemize}

\pagebreak

\section{Исходный код}

Для подготовки корпуса была разработана программа на языке Python, предназначенная для обработки HTML-документов и извлечения текстового содержимого статей. Работа программы организована в виде последовательного конвейера обработки данных.

На первом этапе осуществляется загрузка и разбор HTML-страниц с использованием библиотеки для парсинга HTML-документов. Каждая страница обрабатывается как отдельный документ корпуса.

На следующем этапе выполняется извлечение структурированных элементов статьи: заголовка, информации об авторах и основного текстового содержимого. Для этого используются характерные HTML-теги и атрибуты, специфичные для структуры сайтов Billboard и Pitchfork. Такой подход позволяет отделить полезный текст от вспомогательных элементов страницы.

После извлечения текст очищается от лишних пробелов и объединяется в единое текстовое представление документа. Полученные документы сохраняются для дальнейшего использования в задачах информационного поиска, а также используются для подсчёта статистических характеристик корпуса.

Результатом работы программы является структурированный корпус текстовых документов, готовый для индексации, анализа и применения различных методов поиска и ранжирования.
