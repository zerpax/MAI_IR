\section{Выводы}

В ходе выполнения лабораторной работы по курсу я получил практический опыт разработки поискового робота для автоматической обкачки веб-документов. Была изучена архитектура простого веб-краулера и основные принципы его работы.

В процессе реализации стало очевидно, что важной задачей является корректное управление состоянием обхода, особенно в условиях возможного прерывания работы программы. Использование базы данных и конфигурационного файла позволило обеспечить возобновление работы без потери уже собранных данных.

Также был получен опыт реализации механизма переобкачки документов и проверки изменений содержимого страниц. Это позволило лучше понять проблемы актуальности данных и способы их решения при работе с веб-источниками.

Полученные знания и навыки будут полезны при дальнейшем изучении информационного поиска, разработке собственных поисковых систем, а также при работе с большими объёмами веб-данных и распределёнными источниками информации.
