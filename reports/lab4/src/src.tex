\section{Описание}

\subsection{Общая архитектура индекса}

В работе реализован булев инвертированный индекс, в основе которого лежит отображение термина
в список идентификаторов документов, содержащих данный термин.
Индекс реализован на языке C++ для повышения производительности и интегрирован с Python
с помощью библиотеки \texttt{pybind11}.

Основной структурой данных является хеш-таблица фиксированного размера,
в которой каждый бакет содержит список терминов, попавших в данный хеш.

Каждому термину сопоставляется структура, включающая:
\begin{itemize}
    \item строковое представление термина;
    \item коллекционную частоту термина (collection term frequency);
    \item список идентификаторов документов (posting list).
\end{itemize}

\subsection{Хеширование и хранение данных}

Для распределения терминов по таблице используется полиномиальная хеш-функция с основанием 31.
Размер таблицы задаётся при создании индекса и по умолчанию составляет 100\,000 элементов,
что позволяет снизить количество коллизий при работе с большой коллекцией документов.

При добавлении документа каждый токен:
\begin{itemize}
    \item хешируется;
    \item либо добавляется как новый термин;
    \item либо обновляет существующую запись, увеличивая частоту и дополняя список документов.
\end{itemize}

Добавление идентификаторов документов в posting list происходит с контролем уникальности,
что предотвращает повторное добавление одного и того же документа.

\subsection{Обработка булевых запросов}

Индекс поддерживает выполнение простых булевых запросов следующего вида:
\begin{itemize}
    \item одиночный термин;
    \item \texttt{term1 AND term2};
    \item \texttt{term1 OR term2};
    \item \texttt{NOT term}.
\end{itemize}

Для обработки запросов используются стандартные алгоритмы слияния отсортированных списков:
\begin{itemize}
    \item пересечение списков для операции AND;
    \item объединение списков для операции OR;
    \item дополнение относительно полного множества документов для операции NOT.
\end{itemize}

Все операции выполняются за линейное время относительно размеров posting list,
что является классическим подходом в булевых информационно-поисковых системах.

\subsection{Сериализация индекса}

Для обеспечения повторного использования индекса реализованы методы сохранения и загрузки
в бинарном формате.
В файл записываются:
\begin{itemize}
    \item размер хеш-таблицы;
    \item общее количество документов;
    \item содержимое всех бакетов таблицы, включая термины и posting list.
\end{itemize}

Данный подход позволяет полностью восстановить состояние индекса без повторного анализа корпуса документов.

\subsection{Интеграция с Python}

С использованием библиотеки \texttt{pybind11} класс индекса экспортируется в Python-модуль,
что позволяет:
\begin{itemize}
    \item добавлять документы из Python-кода;
    \item выполнять поисковые запросы;
    \item сохранять и загружать индекс;
    \item использовать индекс совместно с ранее реализованными модулями токенизации и стемминга.
\end{itemize}

Это делает индекс удобным компонентом гибридной системы,
в которой ресурсоёмкие операции выполняются на C++,
а управляющая логика остаётся на Python.
