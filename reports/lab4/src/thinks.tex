\section{Выводы}

В ходе выполнения данной работы были изучены и реализованы основные принципы построения поисковых систем, в частности механизм инвертированного индекса и булевого поиска. В процессе разработки была получена практическая реализация структуры данных, лежащей в основе большинства классических информационно-поисковых систем.

В ходе работы был получен опыт проектирования и реализации инвертированного индекса, включая хранение posting lists, подсчёт частот термов и выполнение логических операций AND, OR и NOT над результатами поиска. Также были изучены алгоритмы эффективного слияния отсортированных списков документов.

Отдельное внимание было уделено сериализации и десериализации сложных структур данных, что позволило реализовать сохранение индекса на диск и его последующую загрузку без повторной обработки документов. В рамках работы был получен практический опыт интеграции C++-кода с Python с использованием библиотеки \texttt{pybind11}, что продемонстрировало возможности комбинирования языков для создания производительных и удобных в использовании решений.

В результате выполнения работы было сформировано целостное понимание процесса индексирования текстовых данных и выполнения поисковых запросов, а также заложена основа для дальнейшего изучения более сложных моделей поиска и ранжирования документов.
