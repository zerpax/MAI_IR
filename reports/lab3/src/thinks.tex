\section{Выводы}

В ходе выполнения работы я научился реализовывать процесс токенизации текста для последующей индексации. Были изучены особенности работы с UTF-8, различными алфавитами и нестандартными символами. На практике было выявлено, что простая токенизация имеет ограничения и требует доработки для корректной обработки сокращений и составных слов.

Кроме того, был внедрён стемминг на основе алгоритма Портера, что позволило объединять словоформы и повысить релевантность поиска. Анализ графика частот токенов и наложение закона Ципфа показало, что редкие термины отклоняются от теоретической прямой из-за низкой статистики и специфики корпуса, что соответствует известным закономерностям языка.

Оценка производительности показала линейную зависимость времени токенизации от объёма текста и скорость обработки, достаточную для небольшого корпуса. Для ускорения обработки возможны параллельная обработка и оптимизация кода.

Полученные навыки будут полезны для дальнейшей разработки поисковой системы, индексации документов и анализа текстовых данных. Реализация токенизации и стемминга позволяет создавать более качественные индексы и повышает точность поиска по различным словоформам.
